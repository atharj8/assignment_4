\documentclass{article}
\usepackage{amsmath}
\begin{document}
\title{ASSIGNMENT 4 }
\author{Athar Javed}
\date{\today}
\maketitle
\flushleft
\textbf{Question:}\\
\hspace{2cm}Find the equation of the line satisfying the following conditions:\\
\begin{enumerate}
\item[a.]passing through the point 
\begin{pmatrix}
-4\\
3\\
\end{pmatrix} with slope 1/2.
\item[b.]passing through the point \begin{pmatrix}
0\\
0
\end{pmatrix} with slope m.
\item[c.]passing through the point \begin{pmatrix}
2\\
2\sqrt{3}
\end{pmatrix} and inclined with the x-axis at an angle of 75◦.

\item[d.] Intersecting the x-axis at a distance of 3 units to the let of the\vspace{0.2cm} origin with slope -2.

\item[e.]Intersecting the y-axis at a distance of 2 units above the origin\vspace{0.2cm} and making an angle of 30◦ with the positive direction of the x-axis.

\item[f.]passing through the points \begin{pmatrix}
-1\\
1
\end{pmatrix} and \begin{pmatrix}
2\\
-4
\end{pmatrix}
\item[g.]perpendicular distance from the origin is 5 and the angle made by\vspace{0.2cm} the perpendicular with the positive x-axis is 30◦.
\end{enumerate}
\newpage
\textbf{SOL:}\\
\textbf{(a).}\hspace{1cm} A=\begin{pmatrix}
-4\\
3
\end{pmatrix} and m=1/2\\
The direction vector is k=\begin{pmatrix}
1\\
1/2
\end{pmatrix}\\
Hence Normal Vector
\begin{center} 
 n=\begin{pmatrix}
0&-1\\
1&0
\end{pmatrix}k\vspace{0.2cm}\\
n=\begin{pmatrix}
0&-1\\
1&0
\end{pmatrix}
\begin{pmatrix}
1\\
1/2
\end{pmatrix}\vspace{0.2cm}\\
n=\begin{pmatrix}
-1/2\\
1
\end{pmatrix}\\
\end{center}
Now, the equation of the line in terms of normal vector in obtained as:\\
\begin{center}
n^T(X-A) = 0\\
\begin{pmatrix}
-1/2&1
\end{pmatrix}X  -  \begin{pmatrix}
-1/2&1
\end{pmatrix}
\begin{pmatrix}
-4\\
3
\end{pmatrix} = 0\\
\textbf{\begin{pmatrix}
-1/2&1
\end{pmatrix}X = 5\\}
\end{center}
which is the required  equation.\\ 
\vspace{1.5cm}
\textbf{(b).}\hspace{1cm} A=\begin{pmatrix}
0\\
0
\end{pmatrix} and slope = m.\\
The direction vector is k=\begin{pmatrix}
1\\
m
\end{pmatrix}\\
Hence Normal vector
\begin{center} 
n=\begin{pmatrix}
0&-1\\
1&0
\end{pmatrix}k\\
\vspace{0.2cm}
n=\begin{pmatrix}
0&-1\\
1&0
\end{pmatrix}
\begin{pmatrix}
1\\
m
\end{pmatrix}\\\vspace{0.2cm}
n=\begin{pmatrix}
-m\\
1
\end{pmatrix}\\
\end{center}
Now, the equation of the line in terms of normal vector in obtained as:\\
\begin{center}
n^T(X-A) = 0\\\vspace{0.2cm}
\begin{pmatrix}
-m&1
\end{pmatrix}X  -  \begin{pmatrix}
-m&1
\end{pmatrix}
\begin{pmatrix}
0\\
0
\end{pmatrix} = 0\\\vspace{0.2cm}
\textbf{\begin{pmatrix}
-m&1
\end{pmatrix}X = 0\\}
\end{center}
which is the required  equation.\\
\vspace{1.5cm}
\textbf{(c).}\hspace{1cm}A=\begin{pmatrix}
2\\
2\sqrt{3}
\end{pmatrix} and slope = tan75◦= 2+ \sqrt{3}\\
The direction vector is k=\begin{pmatrix}
1\\
2+ \sqrt{3}
\end{pmatrix}\\
Hence Normal vector
\begin{center} 
n=\begin{pmatrix}
0&-1\\
1&0
\end{pmatrix}k\\\vspace{0.2cm}
n=\begin{pmatrix}
0&-1\\
1&0
\end{pmatrix}
\begin{pmatrix}
1\\
2+ \sqrt{3}
\end{pmatrix}\\\vspace{0.2cm}
n=\begin{pmatrix}
-2-\sqrt{3}\\
1
\end{pmatrix}\\\vspace{0.2cm}
\end{center}
Now, the equation of the line in terms of normal vector in obtained as:\\
\begin{center}
n^T(X-A) = 0\\
\begin{pmatrix}
-2-\sqrt{3}&1
\end{pmatrix}X  -  \begin{pmatrix}
-2-\sqrt{3}&1
\end{pmatrix}
\begin{pmatrix}
2\\
2\sqrt{3}
\end{pmatrix} = 0\\
\textbf{\begin{pmatrix}
-2-\sqrt{3}&1
\end{pmatrix}X = -4\\}
\end{center}
which is the required  equation.\\
\vspace{1.5cm}
\textbf{(d).}\hspace{1cm}A=\begin{pmatrix}
-3\\
0
\end{pmatrix} and slope = -2\\
The direction vector is k=\begin{pmatrix}
1\\
-2
\end{pmatrix}\\
Hence Normal vector
\begin{center} 
n=\begin{pmatrix}
0&-1\\
1&0
\end{pmatrix}k\\\vspace{0.2cm}
n=\begin{pmatrix}
0&-1\\
1&0
\end{pmatrix}
\begin{pmatrix}
1\\
-2
\end{pmatrix}\\\vspace{0.2cm}
n=\begin{pmatrix}
2\\
1
\end{pmatrix}\\
\end{center}
Now, the equation of the line in terms of normal vector in obtained as:\\
\begin{center}
n^T(X-A) = 0\\
\begin{pmatrix}
2&1
\end{pmatrix}X  -  \begin{pmatrix}
2&1
\end{pmatrix}
\begin{pmatrix}
-3\\
0
\end{pmatrix} = 0\\
\textbf{\begin{pmatrix}
2&1
\end{pmatrix}X = -6\\}
\end{center}
which is the required  equation.\\
\vspace{01.5cm}
\textbf{(e).}\hspace{1cm}A=\begin{pmatrix}
0\\
2
\end{pmatrix} and slope = tan30◦= 1\sqrt{3}\\
The direction vector is k=\begin{pmatrix}
1\\
1\sqrt{3}
\end{pmatrix}\\
Hence Normal vector
\begin{center} 
n=\begin{pmatrix}
0&-1\\
1&0
\end{pmatrix}k\\\vspace{0.2cm}
n=\begin{pmatrix}
0&-1\\
1&0
\end{pmatrix}
\begin{pmatrix}
1\\
1\sqrt{3}
\end{pmatrix}\\\vspace{0.2cm}
n=\begin{pmatrix}
-1\sqrt{3}\\
1
\end{pmatrix}\\
\end{center}
Now, the equation of the line in terms of normal vector in obtained as:\\
\begin{center}
n^T(X-A) = 0\\
\begin{pmatrix}
-1\sqrt{3}&1
\end{pmatrix}X  -  \begin{pmatrix}
-1\sqrt{3}&1
\end{pmatrix}
\begin{pmatrix}
0\\
2
\end{pmatrix} = 0\\
\textbf{\begin{pmatrix}
-1\sqrt{3}&1
\end{pmatrix}X = 2\\}
\end{center}
which is the required  equation.\\
\vspace{1.5cm}
\textbf{(f).}\hspace{1cm} A=\begin{pmatrix}
-1\\
1
\end{pmatrix} and B=\begin{pmatrix}
2\\
-4
\end{pmatrix}\\
\vspace{0.2cm}
The equation of the line joining the points A and B is obtained by:\\
\begin{center}
\textbf{X = A + \lambda(B-A)}\\
\end{center}
Therefore,\\
\begin{center}
X = \begin{pmatrix}
-1\\
1
\end{pmatrix}+ \lambda \begin{pmatrix}
3\\
-5
\end{pmatrix}\\
\end{center}
which is the required equation\\
\vspace{1.5cm}
\textbf{(g).}\hspace{1cm} The perpendicular P intersects the lines M and L at the foot\vspace{0.2cm} of perpendicular.Thus,\\
\begin{center}
n^TP=c\\\vspace{0.2cm}
P=A+\lambda n\\\vspace{0.2cm}
or n^TP = n^TA + \lambda \vert\vert n \vert\vert ^2 = c \\\vspace{0.2cm}
-\lambda = \frac{n^TA - c}{\vert\vert n \vert \vert ^2}\\

\end{center}
Also, the distance between A and L is obtained from:\\
\begin{center}
P = A + \lambda n\\\vspace{0.2cm}
\vert\vert P - A\vert\vert = \vert \lambda\vert \vert\vert n\vert\vert\\ 
\end{center}
Also, n= \begin{pmatrix}
1\\
tan30◦
\end{pmatrix}, A=0\\\vspace{0.2cm}
Thus equation becomes \vspace{0.2cm}
\begin{center}
5 = \vert c\vert / \vert \vert n \vert \vert\\ \vspace{0.2cm}
c = \pm 5 \sqrt{1 + tan^230◦}\\\vspace{0.2cm}
c = \pm 5sec30◦\\
\end{center} 
Thus the required equation is \\
\begin{center}
\begin{pmatrix}
1& 1/\sqrt{3}
\end{pmatrix}\textbf{c} = \pm 10\sqrt{3}\\
\end{center}




\end{document}