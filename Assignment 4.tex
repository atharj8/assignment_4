\documentclass{article}
\begin{document}
\title{ASSIGNMENT 4 }
\author{Athar Javed}
\date{\today}
\maketitle
\flushleft
\textbf{Question:}\\
\hspace{2cm}Assume X,Y,Z,W and P are matrices of orders (2 x n), (3 x k),\\
\vspace{0.2cm}
(2 x p),(n x 3) and (p x k) respectively.\\
\vspace{0.5 cm}The restriction on n,k and p so that PY+WY will be defined are:\\
\begin{enumerate}
\item[a.]k=3 and p=n.
\item[b.]k is arbitrary and p=2.
\item[c.]p is arbitrary and k= 3.
\item[d.]k=2 and p=3.
\end{enumerate}
\vspace{0.2cm}
\textbf{Sol:}\\
\hspace{2cm} We know that\\
\begin{center}
 order of P = (p x k) \\
 \vspace{0.2cm}
 order of y = (3 x k)\\
 \vspace{0.2cm}
 order of W = (n x 3)\\
 
 \end{center}
Therefore for PY to exist,\\
\vspace{0.2cm}
\textbf{The number of columns in matrix P must be equal to number of rows in matrix Y.}\\
\vspace{0.2cm}
Therefore,
$$k = 3$$ \\
Also, number of columns in matrix W = number of rows in matrix Y\\
\vspace{0.2cm}
Therefore, WY will exist.\\
\vspace{0.2cm}
Also,\\
\begin{center}
\vspace{0.2cm}
Order of PY = (p x k)\\
\vspace{0.2cm}
Order of WY = (n x k)\\
\end{center}
\vspace{0.2cm}
Now in order for PY + WY to exist,\\
\vspace{0.2cm}
\textbf{The Order of PY must be equal to order of WY}\\
\vspace{0.2cm}
Therefore \\
\vspace{0.2cm}
\begin{center}
(p x k) = (n x k)\\
\end{center}
\vspace{0.2cm}
which implies that p = n\\
\vspace{0.2cm}
Hence for PY + WY to exist,\\ \center{\textbf{p=n and k=3}}\\
  



\end{document}